\input{../preamble}
\input{../title}
\input{../contents}
\input{../mathequ}
\input{../tables}
\input{../figures}
\input{../float}
\input{../colors}
\begin{document}
\maketitle
\pagenumbering{Roman}
\tableofcontents\newpage
\pagenumbering{arabic}

\setcounter{page}{1}
\section{波和粒子;量子力学基本概念}
每当我们对一个微观体系进行测量时,我们便从根本上干扰了他。这是一种新的性质,因为在宏观领域中我们从来都认为,人们总可以设想出这样的仪器,他们对体系的干扰实际上要多小有多小。\cite{Cohen}
\subsection{电磁波和光子}
\subsection{物质粒子和物质波}%\index[InsertName1]{物质波}
\subsection{对一个粒子的量子描述:波包}
\subsection{在与时间无关的标量势场中的粒子}
\section{量子力学的数学工具}
由路径积分的一些重要性质可以看出,即使被积函数可能是一个复杂的函数,只要被积函数是一个仅包含二阶以下路径变量的指数函数,完全的解便可以求得,可能只差某个相乘因子\cite{Fymman}
\subsection{一个粒子的波函数空间}
\subsection{态空间:狄拉克符号}
\subsection{态空间中的表象}
\subsection{本征值方程,观察算符}%\index[InsertName2]{观察算符}
\subsection{表象和观察算符的两个重要例子}
\subsection{态空间的张量积}
\section{量子力学假定}
能量很低的电子对惰性气体原子的散射,电子几乎不受任何阻拦而完全穿透,这称为Ramsauer-Townsend效应。\cite{Zengshu}
\subsection{可观察量及其测量假定的物理解释}
\subsection{薛定谔方程的物理意义}
\subsection{叠加原理与物理上的预言}
\section{数学符号排版测试}
\subsection{Guass-Bonnet-Theory}
\begin{align}
\oint_C \kappa_g \diff s+\iint_D K \diff \sigma&=2\pi-\sum_{i=1}^n\alpha_i
\end{align}
\subsection{上下积分定义测试}
\begin{align}
\lowint_a^b f(x)\, \diff x&=\inf_P s(p)\\
\overint_a^b f(x)\, \diff x&=\sup_P s(p)
\end{align}
\section{表格排版测试}
\input{./parts/tables_examples}
\section{图形排版测试}
\input{./parts/figures_examples}
\section{浮动体和标题排版测试}
\input{./parts/float_examples}
\section{颜色相关的排版}
%\input{../colors_examples}
\bibliography{References}

\end{document}